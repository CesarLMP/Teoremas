\documentclass{article}
\usepackage{graphicx} 
\usepackage[spanish] {babel}
\usepackage{amsthm}
\usepackage{amssymb}
\theoremstyle{plain}
\newtheorem{theorem} {Teorema} [section]
\newtheorem{corollary}{Corolario}[theorem]
\newtheorem{lemma}[theorem]{Lema}


\title{COM117-202409025}
\author{Cesar Lucas Mamani Posto}
\date{September 2024}

\begin{document}

\maketitle

\section{Teoremas}
\subsection{Ejercicio 1}
\begin{theorem}
Si \(f\) es continua en \([a, b]\) y \(f(a) < c < f(b)\), entonces existe algún \(x\) en \([a, b]\) tal que \(f(x) = c\)
\end{theorem}
\begin{proof}
Sea \(g=f-c\) Entonces \(g\) es continua, y \(g(a)<0<g(b)\). Según el teorema 1, existe algún \(x\) en \([a, b]\) tal que \(g(x) = 0\). Pero esto significa que \(f(x) = c\)
\end{proof}






\end{document}
